\chapter{Justificación}

El presente proyecto tiene como objetivo principal el aplicar todos los conocimientos y habilidades aprendidos en la carrera de Ingeniería Mecatrónica en un dispositivo integral que permita resolver una necesidad actual en la sociedad. \\
Al crear una prótesis estamos aplicando los conocimientos de diseño y manufactura, mientras que al crear un sistema electro-mecánico para el movimiento aplicamos lo aprendido en temas como electrónica, sistemas de control, interfaces humano máquina, entre otras. Actualmente el tema de las prótesis impresas en 3D han tenido un repunte muy alto debido a la disminución de costos a la hora de fabricarlas y la facilidad de obtener diseños y colaborar a través del internet. \\
Si bien estas prótesis son muy útiles, las funciones siguen siendo limitadas por que no incluyen un sistema electrónico para controlarlas, normalmente son sistemas mecánicos básicos que simplemente pueden abrir y cerrar la mano con movimientos muy específicos.  Para resolver esta poca movilidad se han estado realizando investigaciones en diferentes universidades del mundo donde planean controlar las prótesis con un sistema de impulsos cerebrales, por lo que la persona puede moverla simplemente con el pensamiento. \\
Al igual que la impresión 3D, los sistemas de control mediante impulsos cerebrales se han vuelto muy accesibles y fáciles de usar para el usuario común, por lo que ya es posible realizar proyectos con estos dispositivos sin la necesidad de tener un laboratorio especializado en el tema. Para este proyecto se plantea utilizar impulsos básicos para abrir y cerrar la mano al igual que para rotarla.  Como alternativa a esta solución también se plantea que los movimientos se realicen con gestos faciales para un mejor control y que pueda ser utilizado por cualquier persona sin necesidad de entrenarse mentalmente. Si se logra cumplir el objetivo y crear un sistema de control eficiente y fácil de usar, podemos comenzar a implementar nuestro proyecto para organizaciones que se dedican a donar prótesis impresas en 3D, pero gracias a esto, tendrían un mayor uso y devolverían la calidad de vida a todas esas personas afectadas con alguna discapacidad.