%Plantilla basada en "Template for Masters / Doctoral Thesis" (plantilla disponible en writeLaTex) que subió LaTeXTemplates.com

\documentclass[11pt]{book}
\usepackage[paperwidth=17cm, paperheight=22.5cm, bottom=2.5cm, right=2.5cm]{geometry}
\usepackage{amssymb,amsmath,amsthm} %paquete para símbolo matemáticos
\usepackage[spanish]{babel}
\usepackage[utf8]{inputenc} %Paquete para escribir acentos y otros símbolos directamente
\usepackage{enumerate}
\usepackage{graphicx}
%\usepackage{subfig} %para poner subfiguras
\graphicspath{{Img/}} %En qué carpeta están las imágenes
\usepackage[nottoc]{tocbibind}
\usepackage[pdftex,
            pdfauthor={NOMBRE DEL AUTOR},
            pdftitle={TÍTULO DE LA TESIS},
            pdfsubject={ÁREA DE LA TESIS},
            pdfkeywords={PALABRAS CLAVE},
            pdfproducer={Latex con hyperref},
            pdfcreator={pdflatex}]{hyperref}



\begin{document}

%----------------------------------------------------------------------------------------
%	COMANDOS PERSONALIZADOS
%----------------------------------------------------------------------------------------

%SI TU TESIS TIENE TEOREMAS Y DEMOSTRACIONES, PUEDES DESCOMENTAR Y USAR LOS SIGUIENTES COMANDOS

%\renewcommand{\proofname}{Demostración}
%\providecommand{\norm}[1]{\lVert#1\rVert} %Provee el comando para producir una norma.
%\providecommand{\innp}[1]{\langle#1\rangle} 
%\newcommand{\seno}{\mathrm{sen}}
%\newcommand{\diff}{\mathrm{d}}

%\newtheorem{teo}{Teorema}[section] 
%\newtheorem{cor}[teo]{Corolario}
%\newtheorem{lem}[teo]{Lema}

%\theoremstyle{definition}
%\newtheorem{dfn}[teo]{Definición}

%\theoremstyle{remark}
%\newtheorem{obs}[teo]{Observación}

%\allowdisplaybreaks


%----------------------------------------------------------------------------------------
%	PORTADA
%----------------------------------------------------------------------------------------

\title{TÍTULO DE LA TESIS} %Con este nombre se guardará el proyecto en writeLaTex

\begin{titlepage}
\begin{center}

\textsc{\Large Instituto Tecnológico de Estudios Superiores de Monterrey}\\[4em]

%Figura
\begin{figure}[h]
\begin{center}
\includegraphics{logo_ch.jpg}
\end{center}
\end{figure}

\vspace{4em}

\textsc{\huge \textbf{Control de una prótesis de mano robótica mediante un equipo de EEG}}\\[4em]

\textsc{\large Proyecto Final}\\[1em]

\textsc{para aprobar la materia de}\\[1em]

\textsc{Proyecto Integral de Mecatrónica}\\[1em]

\textsc{presenta}\\[1em]

\textsc{\Large Equipo Skywalker}\\[1em]

\textsc{\large Asesor: Carlos Izaguirre}

\end{center}

\vspace*{\fill}
\textsc{Hermosillo, Sonora \hspace*{\fill} 2017}

\end{titlepage}


%----------------------------------------------------------------------------------------
%	Introducción
%----------------------------------------------------------------------------------------

\chapter*{Introducción}

\pagestyle{plain}
%\markboth{PREFACIO23}{PREFACIO} % encabezado 

El presente documento es una investigación que tiene como objetivo evaluar la fatibilidad de elaboración de un proyecto mecatrónico enfocado a la mejora del control del movimiento en prótesis de mano. Las personas que sufren discapacidades severas como daño en la columna vertebral o que sufren de parálisis requieren de dispositivos de asistencia para poder desempeñar actividades cotidianas fundamentales, como alimentarse. Con esta perspectiva en mente, el proyecto busca desarrollar un enfoque relativamente nuevo para aportar una solución viable a esta situación mediante la aplicación de tecnologías de interfaz cerebro-computador. \\
El proyecto consiste en una mano robótica impresa en 3D articulada, cuyo diseño y control permita el movimiento de cada dedo por separado. A diferencia de las prótesis mecánicas o mioeléctricas, el diseño propuesto es mediante el uso de señales electroencefalográficas. Para lograrlo, se utilizará un dispositivo comercial llamado Emotiv Epoc+, que consta de una diadema inalámbrica con conexión bluetooth y 14 canales de transmisión de señales. El dispositivo cuenta con varios tipos de software que permiten realizar estudios sobre la detección de comandos mentales, expresiones faciales y emociones, de tal modo que sea posible generar dispositivos controlados mediante una interfaz cerebro-computadora.\\
Las fuentes obtenidas para la investigación correspondiente son principalmente artículos científicos sobre proyectos similares utilizando EMOTIV EPOC+, de modo que se cuente con antecedentes teóricos y bases prácticas para la realización del proyecto.
El trabajo incluye los objetivos del proyecto, así como su justificación y alcances, además de contar con una sección de marco teórico. 


%----------------------------------------------------------------------------------------
%	TABLA DE CONTENIDOS
%---------------------------------------------------------------------------------------

\tableofcontents


%----------------------------------------------------------------------------------------
%	TESIS
%----------------------------------------------------------------------------------------
\mainmatter %empieza la numeración de las páginas
\pagestyle{headings}

%  Incluye los capítulos en el folder de capítulos

\chapter{Objetivo}

El objetivo de esta prótesis es ofrecer una alternativa funcional y de menor costo para aquellas personas que lo necesiten. Las prótesis con capacidad de movimiento controlados por el usuario existentes en el mercado son muy costosas e invasivas, es por eso que se propone la alternativa de una prótesis de mano controlada por medio de ondas cerebrales gracias al casco EMOTIC EPOC+. Esta opción crea la posibilidad de tener una prótesis manejable, no invasiva y a un menor precio que las disponibles en el mercado.
Este proyecto involucra integración de conocimiento mecánico y electrónico a la hora de diseñar y construir el brazo robótico, así como de programación para el control del brazo y la comunicación con el controlador del casco.
\thispagestyle{empty}
\chapter{Justificación}

El presente proyecto tiene como objetivo principal el aplicar todos los conocimientos y habilidades aprendidos en la carrera de Ingeniería Mecatrónica en un dispositivo integral que permita resolver una necesidad actual en la sociedad. \\
Al crear una prótesis estamos aplicando los conocimientos de diseño y manufactura, mientras que al crear un sistema electro-mecánico para el movimiento aplicamos lo aprendido en temas como electrónica, sistemas de control, interfaces humano máquina, entre otras. Actualmente el tema de las prótesis impresas en 3D han tenido un repunte muy alto debido a la disminución de costos a la hora de fabricarlas y la facilidad de obtener diseños y colaborar a través del internet. \\
Si bien estas prótesis son muy útiles, las funciones siguen siendo limitadas por que no incluyen un sistema electrónico para controlarlas, normalmente son sistemas mecánicos básicos que simplemente pueden abrir y cerrar la mano con movimientos muy específicos.  Para resolver esta poca movilidad se han estado realizando investigaciones en diferentes universidades del mundo donde planean controlar las prótesis con un sistema de impulsos cerebrales, por lo que la persona puede moverla simplemente con el pensamiento. \\
Al igual que la impresión 3D, los sistemas de control mediante impulsos cerebrales se han vuelto muy accesibles y fáciles de usar para el usuario común, por lo que ya es posible realizar proyectos con estos dispositivos sin la necesidad de tener un laboratorio especializado en el tema. Para este proyecto se plantea utilizar impulsos básicos para abrir y cerrar la mano al igual que para rotarla.  Como alternativa a esta solución también se plantea que los movimientos se realicen con gestos faciales para un mejor control y que pueda ser utilizado por cualquier persona sin necesidad de entrenarse mentalmente. Si se logra cumplir el objetivo y crear un sistema de control eficiente y fácil de usar, podemos comenzar a implementar nuestro proyecto para organizaciones que se dedican a donar prótesis impresas en 3D, pero gracias a esto, tendrían un mayor uso y devolverían la calidad de vida a todas esas personas afectadas con alguna discapacidad.
\thispagestyle{empty}
\chapter{Marco Teórico}
Dado que el enfoque central del proyecto es la interpretación de las señales eléctricas generadas por el cerebro, medidas por medio del Emotiv EPOC, es esencial reconocer el trabajo previo que se ha hecho con este sensor. A través de la lectura de diferentes artículos científicos, podemos extraer conocimiento aplicable a nuestro proyecto.

Trabajos previos con este dispositivo, demuestran que es necesario entrenar el software de interpretación a las lecturas únicas de cada individuo, y este proceso resulta ser lento, pero relativamente consistente. Una prueba con sujetos entre 14 y 30 años arrojó que en promedio, este sensor acierta 72.65\% de las veces al interpretar las señales de los usuarios. Esta misma investigación sugiere que la interpretación del EPOC mejora al utilizarse para leer expresiones faciales. Esto abre otras posibilidades de control para nuestro brazo.

Las señales eléctricas emitidas mediante el cerebro son de un orden muy pequeño, por lo que el dispositivo Emotiv EPOC procesa las señales en bruto y las amplifica, convirtiéndolas además a señales digitales que son utilizadas mediante los distintos softwares compatibles. Este proceso interno llevado a cabo por la diadema electrónica facilita la lectura y manipulación de los datos obtenidos, del mismo modo que el software facilita su interpretación. 
Este proceso inicial de adquisición de datos se realiza internamente en el dispositivo EPOC, pero para poder recuperar estas señales en la computadora existe una librería en la interfaz de programación de aplicaciones de Emotiv, mediante la cual es posible comunicar el dispositivo con el software de LabView. 
Es importante establecer un espectro de operación para cada una de las actividades producidas por el usuario, ya sean comandos mentales o movimientos faciales, de modo que puedan registrarse cambios entre actividad neutra y los gestos que deben generar movimiento en la prótesis. En un proyecto previo realizado en 2016 se encontró que para acciones como comandos mentales o ciertas expresiones faciales, la diferencia entre señales era casi imperceptible, sin embargo, para la actividad cerebral generada mediante parpadeos, la diferencia de voltaje era mayor, permitiendo capturar voltajes de origen electroencefalográfico y compararlos con los de origen electromiográfico, siendo mayor este último. Este hallazgo puede aplicarse en el proyecto para facilitar su desarrollo y obtener un control más preciso que al basar los movimientos en comandos mentales. Otra posibilidad es utilizar el giroscopio integrado en el EPOC, pues este presenta una gran respuesta en tiempo real. 















\thispagestyle{empty}


%----------------------------------------------------------------------------------------
%	APÉNDICES
%----------------------------------------------------------------------------------------

\addtocontents{toc}{\vspace{2em}} % Agrega espacios en la toc

\appendix % Los siguientes capítulos son apéndices

%  Incluye los apéndices en el folder de apéndices

%\include{Apendices/AppendixB}
%\include{Apendices/AppendixC}

\addtocontents{toc}{\vspace{2em}} % Agrega espacio en la toc


%----------------------------------------------------------------------------------------
%	BIBLIOGRAFÍA
%----------------------------------------------------------------------------------------
\backmatter
\nocite{*}
\bibliographystyle{plain}
\bibliography{biblio.bib} %Aquí ponen el nombre del archivo .bib




\end{document}